\documentclass{article}
\usepackage[utf8]{inputenc}
\usepackage{fancyhdr}
\usepackage{graphicx}
\usepackage{geometry}

% ---- Commands ------- %
\newcommand{\documentNumber}[1]{
    \LARGE  \textbf{ PUSS214207 } \\
    \medskip
}
\newcommand{\documentVersion}[1]{
    v. {#0.2}

    \medskip
}
\newcommand{\documentTitle}[1]{
    \centerline{\rule{13cm}{0.4pt}}
    \bigskip \bigskip
    \LARGE \textbf{TimeMate} \\
    \bigskip
    \LARGE {Software Specification Document} \\
    \bigskip \bigskip
    \centerline{\rule{13cm}{0.4pt}}
}
\newcommand{\documentGroup}[1]{
    \bigskip \bigskip
    \LARGE Group {#1} \\
    \bigskip
}
\newcommand{\documentResponsible}[1]{
    \LARGE Responsible: {#1} \\
    \medskip
}
\newcommand{\documentAuthors}[1]{
    \LARGE Authors: {#1} \\
    \medskip    
}
\newcommand{\documentDate}[1]{
    \date {#1} 
}

\graphicspath{{./images/}} % Defines a path to file images
\renewcommand{\arraystretch}{1.7}  % Vertical padding for tables


% --- Header & Footer ---- %
\pagestyle{fancy}
\lhead{\leftmark}
\rhead{}
\rfoot{\thepage}
\cfoot{}
\lfoot{}


% ------------------------------------------------ #

% ----- FILL THIS ----- %
\title {
    % Must be 2 digits
    \documentNumber {01}    
    
    % BASELINE.VERSION
    \documentVersion {0.1}
    
    % Full name - SHORTNAME
    \documentTitle {Template}
    \documentGroup {2}
    
    % Options: - Project Management Group
    %          - System Architecture Group
    %          - Developer Group
    %          - Test Group
    \documentResponsible {System Group}
    \documentAuthors {System group}
    
    % Format: YYYY-MM-DD
    \documentDate {2021-03-18}
}

\begin{document}

\maketitle
\thispagestyle{empty}

\newpage

\tableofcontents

\newpage

%---------Document begins here-----------

% FILL IN CORRECT VERSION HISTORY!
% Not sure? Refer to SDP how it works or ask someone!

\section{Document history}

\begin{tabular}{ l | l | l | l }
    Version & Date & Responsible & Description \\
    \hline
    0.1 & 2021-03-03 & SG & Document created. \\
    \hline
    0.2 & 2021-03-18 & SG & Ready for informal review.
   
\end{tabular}

\section{Introduction}
    This document presents the delivered system, and details which version each document is in. It also specifies limitations of the delivered system, and possible differences between the requirements specification and the delivered system.

\section{Delivered system}
 \begin{table}[h]
            \centering
            
             \caption{The delivered system consists of the following documents/parts}
            \begin{tabular}{|l|c|c|c|}
                \hline
                    \textbf{Doc / part} & \textbf{Doc.  number} & \textbf{Version} & \textbf{Comment} \\
                \hline
                    SRS & PUSS214201 & 1.1 &   \\
                 \hline
                    SVVS & PUSS214202 & 1.4 &  \\
                 \hline
                    STLDD & PUSS214204 & 1.0 &  \\
                 \hline
                    SDDD & PUSS214205  & 1.0 & war-file, javadoc and initialization script for database(SQL).   \\
                 \hline
                    SVVI & PUSS214203 & 1.2 &  \\
                 \hline
                    SSD & PUSS214207 & 1.0 &  \\
                 \hline
                    SVVR & PUSS214206 & 1.0 &  \\
                 \hline
                 
            \end{tabular}
           
            \label{activitytable}
        \end{table}

\section{Limitations}


\subsection{System objectives}
The intention for the delivered system is to be used as a web-based time reporting system. The system builds upon the already given \textit{BaseBlockSystem}. The system is specifically created for project groups.

\subsection{Differences compared to the Software Requirements Specification}
Requirement 6.5.2 is implemented differently. Instead of having to press  the button \textit{Add User} to add a new user, the admin is instead presented with the option to add a new user directly under the \textit{User Management} page.



\section{Installation instructions}
In order to install the system, the following is needed.
\begin{enumerate}

    \item System must be started on a machine with a Tomcat server installed.
    
    \item  The system must have access to a MySQL server and the code for the
    jdbc-connection in class ServletBase must be correct. The connection between the MySQL server and the system is already hard coded within the system.
    
    \item A VPN connection has to be established to vpn.lu.se.
    
    \item Activation.jar, javax.mail.jar and servlet-api.jar needs to be added to the class path. These files can be found in \textit{lib} directory which is located in the installation folder of \textit{apache-tomcat-9.0.43}.
    
    \item MySQL must include a database base with a table users, including all roles, as described in STLDD.

\end{enumerate}

%---------Document ends here-----------

\end{document}