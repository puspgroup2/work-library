\documentclass{article}
\usepackage[utf8]{inputenc}
\usepackage{fancyhdr}
\usepackage{graphicx}

% ---- Commands ------- %
\newcommand{\documentNumber}[1]{
    \LARGE  \textbf{ PUSP2142{#1} } \\
    \medskip
}
\newcommand{\documentVersion}[1]{
    v.{#1} \\
    \medskip
}
\newcommand{\documentTitle}[1]{
    \centerline{\rule{13cm}{0.4pt}}
    \bigskip \bigskip
    \LARGE {#1} \\
    \bigskip \bigskip
    \centerline{\rule{13cm}{0.4pt}}
}
\newcommand{\documentGroup}[1]{
    \bigskip \bigskip
    \LARGE Group {#1} \\
    \bigskip
}
\newcommand{\documentResponsible}[1]{
    \LARGE Responsible: {#1} \\
    \medskip
}
\newcommand{\documentAuthors}[1]{
    \LARGE Authors: {#1} \\
    \medskip    
}
\newcommand{\documentDate}[1]{
    \date {#1} 
}

\graphicspath{{./images/}} %Defines a path to file images

% --- Header & Footer ---- %
\pagestyle{fancy}
\lhead{\leftmark}
\rhead{}
\rfoot{\thepage}
\cfoot{}
\lfoot{}


% ------------------------------------------------ #

% ----- FILL THIS ----- %
\title {
    % Must be 2 digits
    \documentNumber {01}    
    
    \documentVersion {0.1}
    
    % Full name - SHORTNAME
    \documentTitle {Template}
    \documentGroup {2}
    
    % Options: - Project management Group
    %          - System architecture Group
    %          - Developer Group
    %          - Test Group
    \documentResponsible {Project management Group}
    \documentAuthors {Project management group}
    
    % Format: YYYY-MM-DD
    \documentDate {2021-01-25}
}

\begin{document}

\maketitle
\thispagestyle{empty}

\newpage

\tableofcontents

\newpage



\section{Introduction}

This document presents requirements for the (Sätt in namn). (Sätt in namn) is a system which purpose and main functionality is to administer time reporting with web-usage capabilities. 


\section{Reference documents}

\begin{enumerate}
  \item Software Requirements Specification: BaseBlockSystem, v. 1.0, Doc. number: PUSS12002
  \item The requirements on the usability of the system applies to this document and are therefore implemented in “System-namn”, following requirements are implemented; 6.1.1, 6.1.2, 6.1.3, 6.1.4, 6.1.6, 6.1.8, 6.1.9, 6.2.2
\end{enumerate}

\section{Background and goals}
\subsection{Main goals}

The goal is to develop and distribute a web-based system where the user can report time and administrate the system according to their roles.

\subsection{Actors and their objectives}
The system can be used and administrated by following actors;

\begin{itemize}
  \item \textbf{User:} The user has the authority to log in to the system to report and  change past reported time as well as review their reported times. The user also inherits roles as either "SG", "UG", and "TG".
  \item \textbf{Project Leader:}
  The project leaders main objective is to administer groups within the project, this implies that the project leader has the authority to add and remove as well as assign users from/to designated roles.
   \item \textbf{Administrator:} The administrator has the authority to add and remove users from the system along with creating and removing project groups. The admin is the only role that can assign the role “Project leader” to a user. The main goal of this role is to be able to administrate creation and removal of users.
\end{itemize}

\includegraphics[width=0.5\textwidth]{Testbild.png}

\caption{Image is to complex to understand}

\section{Terminology}

\section{Context diagram}

\section{Functional requirements}
\subsection{Login and logout}
\subsubsection{Requirement (SÄTT NUMMER)}
"TEST TEST TEST .. TEST .. TEST. TEST TEST TEST TEST TEST .. TEST .. TEST. TEST TESTTEST TEST TEST .. TEST .. TEST. TEST TESTTEST TEST TEST .. TEST .. TEST. TEST TEST
"TEST TEST TEST .. TEST .. TEST. TEST TEST TEST TEST TEST .. TEST .. TEST. TEST TESTTEST TEST TEST .. TEST .. TEST. TEST TESTTEST TEST TEST .. TEST .. TEST. TEST TEST
\subsubsection{Requirement (SÄTT NUMMER}
"TEST TEST TEST .. TEST .. TEST. TEST TEST TEST TEST TEST .. TEST .. TEST. TEST TESTTEST TEST TEST .. TEST .. TEST. TEST TESTTEST TEST TEST .. TEST .. TEST. TEST TEST"
\subsubsection{Requirement (SÄTT NUMMER}
"TEST TEST TEST .. TEST .. TEST. TEST TEST TEST TEST TEST .. TEST .. TEST. TEST TESTTEST TEST TEST .. TEST .. TEST. TEST TESTTEST TEST TEST .. TEST .. TEST. TEST TEST"
\subsubsection{Requirement (SÄTT NUMMER}
"TEST TEST TEST .. TEST .. TEST. TEST TEST TEST TEST TEST .. TEST .. TEST. TEST TESTTEST TEST TEST .. TEST .. TEST. TEST TESTTEST TEST TEST .. TEST .. TEST. TEST TEST"
\subsubsection{Requirement (SÄTT NUMMER}
"TEST TEST TEST .. TEST .. TEST. TEST TEST TEST TEST TEST .. TEST .. TEST. TEST TESTTEST TEST TEST .. TEST .. TEST. TEST TESTTEST TEST TEST .. TEST .. TEST. TEST TEST"
\subsubsection{Requirement (SÄTT NUMMER}
"TEST TEST TEST .. TEST .. TEST. TEST TEST TEST TEST TEST .. TEST .. TEST. TEST TESTTEST TEST TEST .. TEST .. TEST. TEST TESTTEST TEST TEST .. TEST .. TEST. TEST TEST"


\subsection{Data}
\subsubsection{Requirement (SÄTT NUMMER}
\subsubsection{Requirement (SÄTT NUMMER}
\subsubsection{Requirement (SÄTT NUMMER} 

\subsection{User}
\subsubsection{Requirement 6.5.1}
A user can create and submit a time report to the project group which they are assigned to.
\subsubsection{Requirement 6.5.2}
The following scenario should be supported by the system;\\

\textbf{Scenario:} The user wants to submit a time report

\textbf{Prerequisites} The user is logged in to the system

\begin{enumerate}


\item The user navigates to the page “Time Report”. 
\item The user is presented with an overview of their previously reported time. 
\item The user presses the button “Create New Time Report”
\item The user fills in the blank textboxes with the time they have spent on the different activities.
\item The user presses the button for “Submit”.
\item The user is presented with a popup window confirming that the changes have been made.
\item An updated view of the page is displayed.
\end{enumerate}

\subsubsection{Requirement 6.5.3}
A user should be able to see a summarized view of previous reported time.

\subsubsection{Requirement 6.5.4}
The following scenario should be supported by the system;\\

\textbf{Scenario:} The user wants to see a summary of their reported time.

\textbf{Prerequisites} The user is logged in to the system
\begin{enumerate}


\item The user navigates to the page “Time report”.
\item A page is displayed which includes the option to report time, edit old time reports but also a summarized view of their previously logged time. 
\end{enumerate}

\subsubsection{Requirement 6.5.5}
A user should be able to change their time report after it has been submitted to the system.
\subsubsection{Requirement 6.5.6}
A user should be able to delete a previously submitted time report.
\subsubsection{Requirement 6.5.7}
The following scenario should be supported by the system;\\

\textbf{Scenario:} A user wants to change or delete previously reported time.

\textbf{Prerequisites} The user is logged in to the system

\begin{enumerate}

\item The user navigates to the page “Time report”.
\item The user is presented with an overview of their previously reported time. 
\item The user presses the button “Edit time report”
\item The user chooses a time report.
\item The user is presented with a new page which includes the previously reported time table and a button for “Delete”. 
\item The user adds/fills in the new time or chooses to delete the report.
\item The user clicks on the button “Submit Change”.
\item The user is presented with a popup window confirming that the changes have been made.  
\item An updated view of the page is displayed.

\end{enumerate}
\subsubsection{Requirement 6.5.8}
A user should be able to change their password.

\subsubsection{Requirement 6.5.9}
The following scenario should be supported by the system;\\

\textbf{Scenario:} A user wants to change their password.

\textbf{Prerequisites} The user is logged in to the system

\begin{enumerate}
    \item The user clicks on the “Change Password” button in the menu.
    \item The user is presented with a page which includes three relevant textboxes.
    \item The user fills in the first textbox with their current password.
    \item The user fills in the second textbox with their new password.
    \item The user repeats their new password in the third textbox.
    \item The user clicks the button “Change Password”.
    \item The user's password is changed in the system.
    \item The user is presented with a popup window confirming that the changes has been made.
\end{enumerate}

\subsubsection{Requirement 6.6.0}
The following scenario should be supported by the system;\\

\textbf{Scenario:} A user wants to change their password with invalid entry.

\textbf{Prerequisites} The user is logged in to the system

\begin{enumerate}
    \item The user clicks on the “Change Password” button in the menu.
    \item The user is presented with a page which includes three relevant textboxes.
    \item The user fills in the first textbox with their current password.
    \item The user fills in the second textbox with their new password which does not meet the requirement 6.2.3 in PUSS12002.
    \item The user repeats their new password in the third textbox.
    \item The user clicks the button “Change Password”.
    \item The user's password is not changed and a popup window with an error message is displayed.
    \item The user is sent back to stage 3.
\end{enumerate}

\subsection{Project Leader}
\subsubsection{Requirement (SÄTT NUMMER}
\subsubsection{Requirement (SÄTT NUMMER}
\subsubsection{Requirement (SÄTT NUMMER}



\section{Quality Requirements}
\subsection{Maintainability}
\subsubsection{Requirement (SÄTT NUMMER}
\subsubsection{Requirement (SÄTT NUMMER}
\subsubsection{Requirement (SÄTT NUMMER}



\section{Project requirements}
\subsection{Development environment}
\subsubsection{Requirement (SÄTT NUMMER}
\subsubsection{Requirement (SÄTT NUMMER}
\subsubsection{Requirement (SÄTT NUMMER}

\end{document}
